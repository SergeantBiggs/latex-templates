\documentclass[a4paper]{article}
\usepackage[utf8]{inputenc}
\usepackage{hyperref}
\usepackage{fancyhdr}
\usepackage{csquotes}
\usepackage{tabularx}
\usepackage{booktabs}

% add bibliography to TOC
\usepackage{tocbibind}

% German specific stuff
\usepackage[ngerman]{babel}
% Hyphenation for german
\usepackage{hyphenat}
\hyphenation{Mathe-matik wieder-gewinnen}

% To create dummy text
\usepackage{lipsum}

% Citation + Bibtex
\usepackage[
    backend=biber,
    style=numeric,
    citestyle=numeric,
]{biblatex}

\addbibresource{bibliography.bib}

\hypersetup{
    colorlinks=true,
}

% Fancyhdr
\pagestyle{fancy}
\fancyhead[L]{Abschlussdokumentation IHK}
\fancyfoot[L]{J. Reinert}
\fancyfoot[R]{\today}

% Article stuff
\title{Abschlussdokumentation IHK}
\author{Johannes Reinert}
\date{\today}


\begin{document}
\maketitle
\newpage

\tableofcontents

\newpage

\section{Einleitung}
\lipsum{1}
\subsection{Punkt 1}

Testing the citation \cite{test} tool.

\section{Projektbeschreibung}
\lipsum{2}

\section{Projektplanung}
\lipsum{3}

\section{Projektdurchführung}

\section{Abschluss}

\section{Anhang}

\subsection{Glossar}
    \begin{tabularx}{\textwidth}{
            %|| >{\raggedright\arraybackslash}X 
            %| >{\raggedright\arraybackslash} X || }
            || X | X ||}
        \hline

        Begriff & Definition \\
        \hline\hline

        Zeroth law of thermodynamics & If two thermodynamic systems are each in thermal equilibrium with
        a third one, then they are in thermal equilibrium with each other. \\
        \hline

    \end{tabularx}

\subsection{Bibliographie}
\printbibliography

\end{document}