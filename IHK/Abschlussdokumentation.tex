\documentclass[a4paper]{article}
\usepackage[utf8]{inputenc}
\usepackage{hyperref}
\usepackage{fancyhdr}
\usepackage{csquotes}
\usepackage{tabularx}
\usepackage{booktabs}
\usepackage{graphicx}

% add bibliography to TOC
\usepackage{tocbibind}

% German specific stuff
\usepackage[ngerman]{babel}
% Hyphenation for german
\usepackage{hyphenat}
\hyphenation{Mathe-matik wieder-gewinnen}

% To create dummy text
\usepackage{lipsum}

% Citation + Bibtex
\usepackage[
    backend=biber,
    style=numeric,
    citestyle=numeric,
]{biblatex}

\addbibresource{bibliography.bib}

\hypersetup{
    colorlinks=true,
}

% Fancyhdr
\pagestyle{fancy}
\fancyhead[L]{Abschlussdokumentation IHK}
\fancyfoot[L]{A. Author}
\fancyfoot[R]{\today}

\begin{document}

\begin{titlepage}
    \begin{center}
        \Huge
        \textbf{Abschlussdokumentation IHK}

        \vspace{0.5cm}
        \LARGE
        Untertitel \\
        \today

        \vspace{1.5cm}

        \textbf{Author Name} \\
        Musterstraße 100 \\
        1000 Berlin \\
        Identnummer: 000000000 \\
        Prüfungsauschuss: Foo\_auschuss

        \vfill

        Abschlussprüfung \\
        Fachinformatiker für Systemintegration.

        \vspace{0.8cm}

        \includegraphics[width=0.4\textwidth]{fu_logo.png}

        \Large
        Freie Universität Berlin\\
        Kaiserswerther Straße 16-18 \\
        14195 Berlin \\

    \end{center}
\end{titlepage}

\newpage

\tableofcontents

\newpage

\section{Einleitung}
\lipsum{}
\subsection{Punkt 1}

Testing the citation \cite{test} tool.

\section{Projektbeschreibung}
\lipsum{}

\section{Projektplanung}
\lipsum{}

\section{Projektdurchführung}

\section{Abschluss}

\section{Anhang}

\subsection{Glossar}
    \begin{tabularx}{\textwidth}{
            || X | X ||}
        \hline

        Begriff & Definition \\
        \hline\hline

        Zeroth law of thermodynamics & If two thermodynamic systems are each in thermal equilibrium with
        a third one, then they are in thermal equilibrium with each other. \\
        \hline

    \end{tabularx}

%\subsection{Bibliographie}
\printbibliography[heading=bibintoc]

\end{document}